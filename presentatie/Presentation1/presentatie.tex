Dag iedereen...


Ik heb nog geen sluitende probleemstelling aangezien het misschien nog wat te vroeg is om me aan één specifieke hypothese te binden, maar het is me wel al duidelijk hoe ik het onderwerp van mijn thesis zal aanpakken en ik zal nu dus zeer kort beschrijven waar ik me momenteel reeds mee bezig houdt en welke taak me nog rest.

De inhoud van mn thesis kun je losweg omschrijven als mining adverse events from healthcare data. Die mining slaat hier uiteraard op data mining, en hetgeen we hopen uit die data te halen zijn dus die adverse events. Vrij vertaald betekend adverse event simpelweg 'bijwerking' en er zijn veel definities die er aan worden gegegeven, maar ik kies diegene die gegeven werd door het Institute for healthcare improvement. { - zie slide - } 

Ik beschik dus over data uit verschillende ziekenhuizen met allerlei tabellen en bijhorende attributen. Dit is een real-world data set en ik werk hier dus voor samen met Agfa Healthcare. Adverse events worden uiteraard niet expliciet vermeld dus we hopen met data mining deze zaken te kunnen extraheren. { - vermeld nut -} 

Deze data komt dus uit verschillende databases en alhoewel we perfect deze data zouden kunnen extraheren via rechtstreekse SQL-queries op de database, { - ga naar volgende slide -}  hebben mijn begeleider bij Agfa en ik er in overleg voor gekozen om eerst een mapping te doen naar een meer algemeen niveau, zo kunnen we na zo'n mapping over eender welke database spreken binnen één specifieke ontologie. Dit is behoorlijk nuttig aangezien databases onderling behoorlijk verschillen aangezien er weinig standardisatie is,... { - leg omzetting wat verder uit VERMELD dat gebruik gemaakt wordt van technieken SEMANTIC WEB -} 

Vervolgens nog andere standaard data pre-processing taken zoals het eventuele opkuisen van data en een sanity check om te verzekeren tegen invoer fouten etc.

Nadat deze stap doorlopen is zal het erop aankomen het echte data mining uit te voeren. De adverse events zijn impliciet aanwezig in de data en worden momenteel door kleine teams van dokters en verplegers geextraheerd. Het doel is dus om dit te automatiseren. Een eerst ongemak is dat we niet beschikken over een validatie-set, Ik beschik wel over een dokter die de uitvoer van een algoritme zou kunnen controleren of een paar patiënten records zou kunnen labelen, maar het is niet bepaald realistisch om hem te vragen om een volwaardige validatieset op te stellen. Bijgevolg zal ik wss aangewezen zijn op active learning technieken... Ik zal ook multi-relationele technieken dienen te gebruiken aangezien de data verspreid is over heel wat tabellen en we die uiteraard niet kunnen joinen zonder informatieverlies. Het is uiteraard belangrijk hier al wat over te denken, maar dit is uiteraard momenteel nog niet mijn grootste zorg, dus hier ga ik nu momenteel niet dieper op in.

De vraag die nu nog overblijft is: hoe meet ik succes? Wanneer ben ik in mn opzet geslaagd?
Wel, aangezien we de werkbelasting van die teampjes dokters en verplegers wensen te verlagen is succes in de eerste plaats bereikt als we deze detectie kunnen automatiseren en dus met een gelijkwaardige kwaliteit adverse events kunnen detecteren in de patientenrecords. Er zijn momenteel al heel wat adverse event triggers gedefinieerd. Dit zijn dus een reeks richtlijnen en 'symptomen' die tijdens de manuele controle kunnen helpen bij het detecteren van die adverse events. We hopen uiteraard zelf dergelijke adverse event triggers te genereren tijdens het data minen, en een tweede measure of succes zou dus kunnen zijn indien we er in slagen triggers nuttige, geldige triggers te ontdekken die voorheen nog onbekend waren.