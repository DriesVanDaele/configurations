\documentclass{beamer} 
\usepackage{algorithm}
\usepackage{adjustbox}
\usepackage[noend]{algorithmic}
\usepackage{fancybox}
\floatname{algorithm}{Procedure}
\renewcommand{\algorithmicrequire}{\textbf{Input:}}
\renewcommand{\algorithmicensure}{\textbf{Output:}}
\usepackage{wrapfig}
\setbeamertemplate{navigation symbols}{}
\usetheme[height=0mm]{Rochester}
\useinnertheme{rectangles}
\newcommand{\LINEIF}[2]{%
    \STATE\algorithmicif\ {#1}\ \algorithmicthen\ {#2} \ %algorithmicend\ \algorithmicif%
}

\title[Master thesis: presentation]
{Master thesis: presentation}
\author[Dries Van Daele]{Dries Van Daele}
\institute{DTAI masterproef KU Leuven}
\date{6 september, 2013}

\begin{document}

\begin{frame}{}
  \titlepage
\end{frame}

\begin{frame}{Methodology}
  \begin{table}
    \centering
    \adjustbox{max height=\dimexpr\textheight/3\relax,
               max width=\textwidth}{
      \begin{tabular}{| l | l | l | l |}
        \hline
        patient\_id & birth\_date  & gender \\ \hline
        patient\_1  & 12-AUG-1956 & M \\
        patient\_2  & 21-FEB-1973 & F \\
        patient\_3  & 30-AUG-1989 & M \\
        \hline
      \end{tabular}}
    \caption {a relational database table storing patient data}
    \label{table1}
  \end{table}

  \begin{table}
    \centering
    \adjustbox{max height=\dimexpr\textheight-9cm,
               max width=\textwidth}{
      
      \begin{tabular}{| l | l | l | l|}
        \hline
        medical\_case\_id  & patient\_id & admission\_date & discharge\_date \\ \hline
        medical\_case\_100 & patient\_1  & 17-JAN-2009    & 19-JAN-2009    \\
        medical\_case\_101 & patient\_1  & 03-SEP-2009    & 27-SEP-2009    \\
        medical\_case\_102 & patient\_2  & 06-NOV-2010    & 13-NOV-2010    \\
        medical\_case\_103 & patient\_3  & 21-DEC-2010    & 03-JAN-2011    \\
        medical\_case\_104 & patient\_3  & 28-MAR-2012    & 02-JUN-2012    \\
        \hline
      \end{tabular}}
      \caption {a relational database table storing medical case data}
      \label{table2}
    \end{table}

    \begin{table}
      \centering
    \adjustbox{max height=\dimexpr\textheight/3\relax,
               max width=\textwidth}{

      \begin{tabular}{| l | l | l |}
        \hline
        diagnose\_id & medical\_case\_id  & ICD\_code \\ \hline
        diagnose\_1  & medical\_case\_101 & F48      \\
        diagnose\_2  & medical\_case\_101 & T85.4    \\
        diagnose\_3  & medical\_case\_103 & F41.0    \\
        diagnose\_4  & medical\_case\_103 & F48.0    \\
        diagnose\_5  & medical\_case\_103 & R55      \\
        \hline
      \end{tabular}}
    \caption {a relational database table storing diagnosis data}
    \label{table3}
  \end{table}
  

\end{frame}
% 




% % \begin{lstlisting}[style=rdf,float, caption={Two rules in N3, one imposes a space:containedBy relation, the other expresses the transitive closure over this relation}, label=rule_example]
% % @prefix orgStructure: <http://www.example.org/database/ddo/orgStructure#> .
% % @prefix space: <http://eulersharp.sourceforge.net/2003/03swap/space#> .

% % {
% %         _:structure 
% %                 orgStructure:structID ?s ;
% %                 orgStructure:innerStructID ?is .
% % } => {
% %         ?is space:containedBy ?s .  
% % } .

% % {
% %         ?startNode space:containedBy ?middleNode .
% %         ?middleNode space:containedBy ?endNode .
% % } => {
% %         ?startNode space:containedBy ?endNode .
% % } .
% % \end{lstlisting}

% \end{frame}
\end{document}